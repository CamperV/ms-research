\clearpage
\begin{centering}
\textbf{SUMMARY}\\
\vspace{\baselineskip}
\end{centering}

The motivation for this research is to provide a quantitative model for improving shadow detection in arbitrary environments. Many computer vision applications extract foreground pixels to capture moving objects in a scene. However, since shadows share movement patterns with foreground objects (and have a similar magnitude of intensity change compared to a background model), they tend to be extracted with the desired object pixels. Shadows generally contribute to inaccurate object classifications and impede proper tracking of foreground objects. Contemporary shadow removal algorithms have made great strides in discriminating between object pixels and shadow pixels, but lack scene-independence. In order to perform optimally, these leading methods require assumptions to be made about key factors of a scene, including illumination constancy, color content, and shadow intensity. As a result, no leading shadow removal method is robust enough to compensate for environmental change over time, nor are they suitable for deployment into a particular environment without a priori tuning of parameters.

This research evaluates popular shadow removal methods, extracts corresponding algorithmic parameters that affect shadow removal, correlates these parameters with salient environmental aspects, and finally leverages this correlation to improve shadow removal efficacy across diverse environments. Data collection and validation were performed using a collection of widely-used computer vision datasets. Parameters, both algorithmic and environmental in nature, are identified, correlated, and evaluated using analytic tools. 
%Particularly, a strong correlation is found using the average attenuation of dark foreground pixels in a frame. 
Using the average attenuation of dark foreground pixels, an adaptive model improves shadow detection by up to 10\% and improves shadow-object discrimination by up to 28\%. Additional indirect environmental factors are found to increase the effectiveness of this attenuation model. 
%Measuring the color bias introduced by shadows in a scene allows for the selection of an appropriate brightness-attenuation model per environment, boosting correlation by 7\% to 20\%.
Brightness calculation methods are shown to improve attenuation correlation by 7\% to 20\%.
Identifying low-contrast feature keypoints in a scene is also found to improve attenuation-correlation in some environments by up to 12\%.

%\pagenumbering{gobble}  %remove page number on summary page
