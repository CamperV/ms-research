\documentclass[12pt]{report}
\usepackage{setspace}
\usepackage{outlines}
\usepackage{enumitem}
\setenumerate[1]{label=\Roman*.}
\setenumerate[2]{label=\Alph*.}
\setenumerate[3]{label=\roman*.}
\setenumerate[4]{label=\alph*.}

\begin{document}
\begin{outline}[enumerate]

\1 Abstract

\1 Chapter 1 - Introduction

\1 Chapter 2 - Background \& Motivation
	\2{Shadow Removal}
		\3{Common Applications/Use-Cases}
			\4{Detection/Discrimination}
		\3{Shadow Detection Methods}
			\4{Chromacity}
			\4{Physical}
			\4{Geometry}
			\4{SRT}
			\4{LRT}
		\3{The Problem}
			\4{\textit{"In order to perform optimally, these leading [shadow removal] methods require assumptions to be made about key factors of a scene, including illumination constancy, color content, and shadow intensity. As a result, no leading shadow removal method is robust enough to compensate for environmental change over time, nor are they suitable for deployment into a particular environment without a priori tuning of parameters."}}
	\2{Proposed Research - Objective and Approach}
		\3{Objective}
		\3{Approach}
			\4{Overview - Proof of Concept}
			\4{Evaluation Metrics}

\1 Chapter 3 - Methodology
	\2{Algorithm Assessment}
		\3{Data Collection and Analysis}
			\4{Datasets}
			\4{Graphical Tools}
			\4{SimpleINI}
		\3{Selecting an Algorithm}
			\4{Evaluation of Methods}
		\3{Selecting a Parameter - Physical Shadow Removal}
			\4{Weak Detector - Physical Shadow Removal}
			\4{Strong Detector - Physical Shadow Removal}
			\4{Evaluation of Parameters}
	\2{Environmental Assessment - Environmental Parameters}
		\3{Previous Work - Large Region Texture Removal}
		\3{Attenuation and Saturation}
		\3{Non-linear Attenuation and Spectral Properties of Light}
			\4{Observed Spectral Properties in Outdoor Scenes}
		\3{Brightness Models}
			\4{HSV}
			\4{HSI}
			\4{HSL}
			\4{Relative Luminance (Y)}
			\4{Euclidean Norm}
			\4{HSP}
		\3{Low-contrast SIFT Keypoints}
	\2{Weak Detector Estimation - Creating a Model}

\1 Chapter 4 - Results
	\2{Correlation of Parameters}
		\3{Attenuation Correlation Results}
		\3{Correlation Improvements}
			\4{Low-contrast Keypoints}
			\4{Brightness Models}
	\2{Parameter Model Results}
		\3{Average Attenuation Results}
		\3{Brightness Models}

\1 Chapter 5 - Conclusions

\1 Appendix


\end{outline}
\end{document}
