\documentclass[12pt]{report}
\begin{document}

\clearpage
\begin{centering}
\textbf{SUMMARY}\\
\vspace{\baselineskip}
\end{centering}

\doublespacing

The motivation of this research is to provide a quantitative model for improving shadow detection in arbitrary environments. Many computer vision applications utilize the extraction of foreground pixels to capture moving objects in a scene; however, since shadows share movement patterns with foreground objects (and have a similar magnitude of intensity change compared with the background model), they tend to be extracted alongside the desired object pixels. Shadows generally contribute to inaccurate object classifications and impede proper tracking of foreground objects. Contemporary shadow removal algorithms have made great strides in discriminating between object pixels and shadow pixels, albeit with endemic limitations. In order to perform optimally, these leading methods require assumptions to be made about key factors of a scene, including illumination constancy, color content, clearly defined foreground objects, etc. As a result, no leading shadow removal methods are robust enough to compensate for a scene over time, nor are they suitable for deployment in an environment without a priori tuning of parameters. 
	
The objective of this research is to develop a framework capable of understanding salient scene parameters that affect shadow removal, and implement said shadow removal for an arbitrary scene for an arbitrary length of time.

\pagenumbering{gobble}  %remove page number on summary page
\end{document}
